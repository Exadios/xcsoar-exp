\chapter{Atmosphere and Instruments}\label{cha:atmosph}
XCSoar maintains an internal model of the atmosphere based on
statistics gathered from the flight path and other instruments
connected to the Pocket PC device.  These statistics and measurements
are approximate and the weather can on some days change rapidly.  The
pilot should at all times keep observing the weather.  In
particular, when out-landing in fields, the pilot should look for
indicators on the ground to confirm wind strength and direction.

\section{Variometer}

A needle-dial style display shows the variometer measurements.  The
gross variometer reading drives the main arrow on the dial, and in the
center of the dial the instantaneous measurement is shown as text.
Additionally, speed command arrows (chevrons) appear above or below
the gross variometer measurement.  Chevrons pointing up indicate
slowing down is recommended.  Chevrons pointing down indicates that
speeding up is recommended.  

When the averager value is displayed, the value shown is the average
gross climb rate over the previous 30 seconds when in circling mode,
and the netto (airmass) vertical speed over the previous 30 seconds
when in cruise mode.

\marginpar{\includegraphics[angle=0,width=0.5\linewidth,keepaspectratio='true']{figures/gaugevario2.png}}

The average value can also be displayed as an optional additional
needle (caret).
The vario gauge is customisable \config{variogauge} as to what is displayed
along with the gross value etc.

When an intelligent variometer is connected to XCSoar, the needle
displays data from the instrument; otherwise it produces variometer
estimates based on GPS vertical speed, which is slow and uncompensated
for aircraft total energy.  

The MacCready value, bugs and ballast, optimum speed to fly and wind
data are transferred between XCSoar and supported external intelligent
variometers.  In the ideal setup, both XCSoar and the variometer have
a consistent perspective on the flight at all times; and that by
adjusting the MacCready setting on one device should be kept in sync
with the other, by the software and to not require additional input from
the pilot.

Pilots abuse the device synchronisation (see \ref{conf:comdevices}) for various 
reasons. You may have  different MacCready settings on PDA and inteligent vario 
to cross-check the results. You may do computations with different ballast 
settings and cross-chsck the results. You may choose on one of the devices 
manually a different wind setting and cross-check the results, etc.

A list of supported variometers is maintained in
Section~\ref{sec:supported-varios}.

For Vega: A small icon displaying a circling glider is displayed when
the variometer is in climb audio mode.

\section{Air data inputs}

Where additional aircraft dynamics or air mass data are provided by an
intelligent variometer, XCSoar can often make use of it or display it
in a separate InfoBox.  Key sensor measurements that XCSoar uses include:
\begin{description}
\item[Gross total energy variometer] (rate of change of the total energy of
 the aircraft)  Used for display, and for calculation of netto variometer.
\item[Netto variometer] (estimated vertical velocity of the air mass at
 the aircraft)  Used to for display, and to colour the snail trail
 so that it may effectively show areas of lift and sink.
\item[Aircraft acceleration] (load factor)  Used for netto variometer
  calculations where an external netto variometer is not provided.
\item[Barometric altitude] Used for display
\item[Indicated airspeed] Used for display, in compensating final glide
  calculations for aircraft kinetic energy, and in netto variometer
  calculation where an external netto variometer is not provided.
\item[Air density] Used for calculating true airspeed from indicated
  airspeed.
\end{description}

\section{Wind display}

A continuous display of wind strength and direction is provided on the
map.  The wind information is derived from the gliders wind drift
during thermal flight (climb mode).

The wind direction and speed are displayed as a wind vector on the
moving map display and optionally in numeric form in the data display
fields.  The length of the vector indicates the wind magnitude, and
this magnitude is also displayed near the wind vector.

The wind data is one of many data sources used to calculate final
glide information.  It is possible to manually adjust the wind used in
all calculations.

\begin{center}
\includegraphics[angle=0,width=0.4\linewidth,keepaspectratio='true']{figures/optwind.png}

%{\it DIAGRAM CUTOUT SHOWING WIND VECTOR AT GLIDER, MAYBE ALSO SHOW A
%WINDSOCK NEXT TO THE DISPLAY TO MAKE THE DIRECTION OBVIOUS.  NO
%TERRAIN/TOPOGRAPHY}

\end{center}

\section{Wind estimation}\label{sec:wind-estimation}

XCSoar offers two ways of estimating wind during flight.
\begin{description}
\item[Circling]  This method uses GPS position fixes to estimate the wind
  based on drift, typically while thermalling; and is available on all
  XCSoar installations.
\item[ZigZag]  This method uses GPS position fixes and true airspeed measurements
  to estimate the wind, typically during cruise.  It is only available where
  XCSoar is connected to an intelligent variometer that outputs true airspeed.
\end{description}

The wind magnitude and direction can also be adjusted manually from the
wind settings dialogue (see below).  

Statistics are gathered so that winds are recorded at different
heights and times.  When the glider's altitude changes significantly,
the statistics are consulted to determine the best estimate of the
wind based on previous measurements.

For PC and Pocket PC with touchscreens, you can
also do this by highlighting the wind {\InfoBox} and using the cursor
keys (up and down increase and decrease the magnitude, left and right
rotate the wind direction).

The configuration \config{autowind} settings dialogue allows control of which
estimation method is used for wind updates, via the field `Auto Wind':
\begin{itemize}
\item Manual
\item Circling
\item ZigZag
\item Both (ZigZag and Circling)
\end{itemize}

When wind estimates change significantly, a status message
notification of this is issued.

\subsection*{Circling wind algorithm}

XCSoar estimates the wind magnitude and direction when circling.  It
does this using a sophisticated algorithm that incrementally improves
the wind estimate from completed turns.  Poor quality turns, where
the bank angle changes significantly, are rejected or have minimal
impact on the overall wind estimate.  The best turns are those with
constant bank angle.

Estimates are only obtained if the average GPS fix rate is better than
one every two seconds.  This results in improved fidelity of estimates
in the presence of GPS dropouts.

%{\it DIAGRAM SHOWING WIND CALCULATION, BAD TURN AND GOOD TURN}

\subsection*{Zig-Zag algorithm}

For aircraft fitted with intelligent variometers connected to XCSoar,
a so-called `zig-zag' wind estimation algorithm is available.  With
this algorithm, the wind estimate can be updated continuously during
long glides without circling.

This allows the wind estimate to be updated during cruise while the
aircraft performs a zigzag manoeuver.  No specific manoeuver is
required, in many cases the estimate will be updated as the aircraft's
heading changes naturally as the pilot hunts for lift.  In general,
however, the technique requires the aircraft heading to change over 40
degrees.

If the wind changes significantly while in straight flight, the
zig-zag algorithm is used to update the wind estimate even if the
aircraft's heading does not change much. This provides greater
accuracy in long final glides.

Wind estimates are updated when a large difference between the
estimated ground speed and the true ground speed are detected even
without much zig-zag manoeuvering.

\subsection*{Compass algorithm}

For aircraft fitted with intelligent variometers and digital compasses
connected to XCSoar, a wind estimation algorithm making use of
magnetic heading and airspeed is being developed.  This provides
another method of updating the wind estimate during cruise and does
not require zig-zag manoeuvres.

\section{Wind settings dialogue}\label{sec:wind-setup}

The wind dialogue allows the initial estimate of the wind speed and
direction to be entered, usually prior to flight.
\menulabel{\bmenu{Config 1}\blink\bmenut{Setup}{Wind}}

\begin{center}
\includegraphics[angle=0,width=0.4\linewidth,keepaspectratio='true']{figures/dialog-wind2.png}
\end{center}

At any time during flight, the pilot can make corrections to the wind
estimate by entering the correction in the wind settings dialogue.  Once the
dialogue get closed , the internal estimate is ignored until a new internal
estimate is obtained from the circling or zigzag algorithm.

The automatic wind algorithm may also be switched on or off (or
between modes) in this dialogue.  See Section~\ref{sec:wind-estimation}
for details on these algorithms.

The compensation of wind drift of the snail trail can also be switched
on or off in this dialogue.  See Section~\ref{sec:trail} for details on
how this affects the display of the snail trail.

\section{Thermal profile}

Statistics on climb rates in thermals are collected and displayed in a
thermal band meter.  This is shown above the final glide difference
bar on the left side of the map display.  It is not shown when the
glider is above final glide.  

\vskip 2cm
\marginpar{\hbox{\vbox{\includegraphics[angle=0,width=0.7\linewidth,keepaspectratio='true']{figures/thermalprofile.png}}}}

The thermal band meter shows a graph, where the vertical axis is
height above the break-off height (Section~\ref{sec:safety-heights}), and is
scaled according to the maximum height achieved.  The horizontal axis is the average climb
rate achieved at a particular height band.  The horizontal axis is
scaled according to the MacCready setting, and an arrow indicating this
setting, and the glider's current height is overlaid on the shaded
area.  This scaling and arrow makes it easy to see how the pilot's
MacCready setting compares with achieved thermals and to plan the
desired working height band.

When cruising between thermals, the vertical position of the arrow,
indicating the glider's height relative to the thermal band, can be
used as a reference to suggest how urgent it is to find the next
thermal.  As the arrow approaches the bottom of the band, then the
glider is nearing the break-off height and the pilot should consider
taking even a weak thermal.

\section{Thermal locator}
An algorithm estimates the center of the lift when circling.  The
thermal marker symbol is a green circle with a spiral.

\begin{center}
\includegraphics[angle=0,width=0.8\linewidth,keepaspectratio='true']{figures/shot-tlocator-circling.png}
\end{center}

The thermal locator marks the location of the last 20 
thermals on the map with the thermal symbol during cruise.

This location is calculated to compensate for the thermal drift at 
the glider's height.  This means that internally XCSoar remembers 
the location of the thermal source on the ground.  In other words, 
if you leave a thermal at the top and later return at low altitude, 
the position on the map shows the predicted location of the thermal 
at that low altitude (which is further upwind than the top).  

If the wind changes and the thermal source is still active, its 
position on the map reflects the wind change; that is, the thermal 
at altitude will be projected downwind at the new wind estimate.

\begin{center}
\includegraphics[angle=0,width=0.8\linewidth,keepaspectratio='true']{figures/shot-tlocator-cruise.png}
\end{center}


\section{Thermal assistant}\label{sec:thermal-assistant}

The thermal assistant is a graphical aid to maximise the exploitation of the
given thermal updraft. If it is configured ``On" \config{thermalassistant} the
small polar digram is mapped to the lower left corner of the screen. A
single tap on the small digram enlarges it to a full-screen view. 

The polar diagram shows the climb rate over the circular course of the glider.
The screen-shots show a right circle, where the glider position is fixed to the
left side, and the polar distribution of the climb rate is shown relatively to
the current glider position.


\begin{tabular}{c c}
\includegraphics[angle=0,width=0.5\linewidth,keepaspectratio='true']{figures/dialog-thermal-assistant0.png}&
\includegraphics[angle=0,width=0.5\linewidth,keepaspectratio='true']{figures/dialog-thermal-assistant1.png}\\
\end{tabular}

The to screen-shots are taken in a few seconds sequence to demonstrate the
practical usage of the rotating climb diagram. A simple recipe to optimise
the climb rate according to the assistant would be to follow these two steps
repeatedly:
\begin{description}
\item[1.]  At the moment the maximum peak on the polar diagram passes the top of
the display; that is a quarter of the circle before you reach that part again:
Open the circle a bit to displace the circle center in the direction of the
strongest climb rate.
\item[2.]  At the moment the maximum peak on the polar diagram passes the
gliders position; the vario should show the maximal climb rate: Narrow the
circle as much a spossible to center the thermal updraft at its maximum. 
\end{description}

It must be said, that the interpretation of the thermal assistant allways relays
on the specific lag of the connected sensor and PDA itself. A successful
updraft optimisation will thus depend on a bit training to take the lag into account.


\section{Convection forecast}\label{sec:convection-forecast}

If the glider is equipped with an outside temperature and humidity
probe, a simple convection forecast system estimates the convection
ceiling and the cloud base.  The humidity probe is optional and is
mainly required for estimating cloud base.

Prior to takeoff or during flight the pilot can modify the maximum
forecast temperature on the ground, by adjusting the value in the
``Forecast Temperature'' {\InfoBox}.

The forecast convection ceiling is determined by the altitude at which
the atmospheric temperature equals the maximum forecast temperature on
the ground, cooled adiabatically as it rises according to the dry
adiabatic lapse rate.  Typically the glider will not climb as far as
the convection ceiling and so the measured values are extrapolated to
find the ceiling.  If the atmosphere is stable, the convection ceiling
is reported as zero altitude.

The maximum forecast temperature on the ground is entered using the
flight settings dialogue described in Section~\ref{sec:flight-setup}.


%{\it DIAGRAM SHOWING THERMAL FORECAST, STABLE WITH CLOUD}
%
%{\it DIAGRAM SHOWING THERMAL FORECAST, UNSTABLE }

The forecast cloud base is determined by the altitude at which the dew
point intersects the maximum forecast temperature on the ground,
cooled adiabatically as it rises according to the dry adiabatic lapse
rate.  If no clouds are forecast, the cloud base is reported as zero.

%{\it DIAGRAM SHOWING THERMAL FORECAST, STABLE WITH NO CLOUDS}

\section{Analysis dialogue}

The analysis dialogue is used to see several aspects of the atmosphere.
\menulabel{\bmenu{Info 1}\blink\bmenu{Analysis}}

Several pages of interest:
\begin{description}

\item[Wind at altitude]
  This shows a graph of the wind speed versus height, and shows the
  wind vector at several heights.

The `Set wind' button opens the wind settings dialogue (e.g.\ to
manually set the wind).

\begin{center}
\includegraphics[angle=0,width=0.8\linewidth,keepaspectratio='true']{figures/analysis-wind.png}
\end{center}

\item[Temperature trace]
  This page is only available if a supported instrument is connected
  to XCSoar that produces outside air temperature and humidity.  The
  chart shows the variation of dry air temperature, dew point
  temperature and outside air temperature with height.  The convection
  forecast is summarised as the estimated thermal convection height
  and estimated cloud base.

\begin{center}
\includegraphics[angle=0,width=0.8\linewidth,keepaspectratio='true']{figures/analysis-temptrace.png}
\end{center}

\end{description}
The climb history and barograph pages, described in
Section~\ref{sec:analysis-climb}, are also useful to determine
trends in the soaring conditions.

\section{Weather forecast}\label{sec:weather-forecast}

Weather forecasts, typically generated from RASP (Regional Atmospheric
Soaring Prediction) forecasts, may be overlaid on the map.  The user
must install a `xcsoar-rasp.dat' file, prepared by a RASP provider,
into the XCSoarData directory for this function to be available.

This section of the documentation is intended to describe the basic
functionality; the reader is referred to the RASP website
\url{www.drjack.info} for more details on how RASP forecasts work, from where 
they are available, and their use and limitations.

The forecast overlays are accessed by the weather dialogue. 
\menulabel{\bmenu{Info 2}\blink\bmenu{Weather}}

\begin{center}
\includegraphics[angle=0,width=0.5\linewidth,keepaspectratio='true']{figures/dialog-weather.png}
\end{center}

The Field setting determines which data field is displayed on the map.
The Time setting determines at which forecast time the data field will
be displayed.  Upon entering the weather dialogue, the Time setting is
advanced to the next nearest forecast time available in the RASP file.

When a field is not available in the RASP file, the background is left blank.

The maximum and minimum values of the field in the map area are drawn
at their respective locations on the map.  The field name is displayed
on the lower left of the screen.

The fields available to display are as follows:
\begin{description}
\item[Terrain] Display terrain on map, no weather data displayed.

\item[W*] 
Average dry thermal updraft strength near mid-BL height.  Subtract
glider descent rate to get average vario reading for cloudless
thermals.  Updraft strengths will be stronger than this forecast if
convective clouds are present, since cloud condensation adds buoyancy
aloft (i.e. this neglects ``cloudsuck'').  This value depends upon both
the surface heating and the BL depth.

\begin{center}
\includegraphics[angle=0,width=0.8\linewidth,keepaspectratio='true']{figures/rasp-wstar.png}
\end{center}

\item[BL wind spd] 
The speed and direction of the vector-averaged wind in the BL.  This
prediction can be misleading if there is a large change in wind
direction through the BL.

\begin{center}
\includegraphics[angle=0,width=0.8\linewidth,keepaspectratio='true']{figures/rasp-blwindspd.png}
\end{center}

\item[H bl]  
Height of the top of the mixing layer, which for thermal convection is
the average top of a dry thermal.  Over flat terrain, maximum
thermalling heights will be lower due to the glider descent rate and
other factors.  In the presence of clouds (which release additional
buoyancy aloft, creating ``cloudsuck'') the updraft top will be above
this forecast, but the maximum thermalling height will then be limited
by the cloud base.  Further, when the mixing results from shear
turbulence rather than thermal mixing this parameter is not useful for
glider flying.

\begin{center}
\includegraphics[angle=0,width=0.8\linewidth,keepaspectratio='true']{figures/rasp-hbl.png}
\end{center}

\item[dwcrit]  
This parameter estimates the height above ground at which the average
dry updraft strength drops below 225 fpm and is expected to give
better quantitative numbers for the maximum cloudless thermalling
height than the BL Top height, especially when mixing results from
vertical wind shear rather than thermals.  (Note: the present
assumptions tend to underpredict the max. thermalling height for dry
consitions.) In the presence of clouds the maximum thermalling height
may instead be limited by the cloud base.  Being for ``dry'' thermals,
this parameter omits the effect of ``cloudsuck''.

\item[bl cloud]  
This parameter provides an additional means of evaluating the
formation of clouds within the BL and might be used either in
conjunction with or instead of the other cloud prediction parameters.
It assumes a very simple relationship between cloud cover percentage
and the maximum relative humidity within the BL.  The cloud base
height is not predicted, but is expected to be below the BL Top
height.

\begin{center}
\includegraphics[angle=0,width=0.8\linewidth,keepaspectratio='true']{figures/rasp-blcloudpct.png}
\end{center}

\item[Sfc temp] 
The temperature at a height of 2m above ground level.  This can be
compared to observed surface temperatures as an indication of model
simulation accuracy; e.g. if observed surface temperatures are
significantly below those forecast, then soaring conditions will be
poorer than forecast.
\item[hwcrit]  
This parameter estimates the height at which the average dry updraft
strength drops below 225 fpm and is expected to give better
quantitative numbers for the maximum cloudless thermalling height than
the BL Top height, especially when mixing results from vertical wind
shear rather than thermals.  (Note: the present assumptions tend to
underpredict the max. thermalling height for dry consitions.) In the
presence of clouds the maximum thermalling height may instead be
limited by the cloud base.  Being for ``dry'' thermals, this parameter
omits the effect of ``cloudsuck''.
\item[wblmaxmin]  
Maximum grid-area-averaged extensive upward or downward motion within
the BL as created by horizontal wind convergence. Positive convergence
is associated with local small-scale convergence lines.  Negative
convergence (divergence) produces subsiding vertical motion, creating
low-level inversions which limit thermalling heights.
\item[blcwbase] This parameter estimates the height of the cumulus cloud
base.
\end{description}

\begin{maxipage}
The colour schemes used in rendering the RASP contours are illustrated
in the table below.

\begin{longtable}{c c c}
\includegraphics[angle=0,width=3.5cm,keepaspectratio='true']{figures/ramp-rasp-cloudpct.png}&

\includegraphics[angle=0,width=3.5cm,keepaspectratio='true']{figures/ramp-rasp-h.png}&

\includegraphics[angle=0,width=3.5cm,keepaspectratio='true']{figures/ramp-rasp-temperature.png}\\

\includegraphics[angle=0,width=3.5cm,keepaspectratio='true']{figures/ramp-rasp-vertspeed.png}&
\includegraphics[angle=0,width=3.5cm,keepaspectratio='true']{figures/ramp-rasp-windspeed.png}& \\

\end{longtable}
\end{maxipage}

